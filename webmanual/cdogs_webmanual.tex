% Manual for the CDOGS website.
%
% Written by Clark Richards
% 2009-03-29

\documentclass[letterpaper,12pt]{article}

\usepackage{anysize}
\usepackage{lineno}
\usepackage{graphicx}
\usepackage{fancyhdr}
\usepackage{natbib}
\usepackage{hyperref}

\marginsize{1in}{1in}{0.5in}{0.5in}
\title{C-DOGS Website Manual \\ A description of the myriad of pages,
  scripts, and procedures for a headache-free conference}
\author{Clark Richards}
\date{\today}

\begin{document}
\maketitle
\newpage

\tableofcontents
\newpage

\section{Introduction}

The Conference of Dalhousie Oceanography Graduate Students (or C-DOGS,
as it's come to be known), has existed since 2002 (?). Starting in
2004, a website was created that allowed authors to submit their
abstracts online, easy the job of the organizers by automating much of
the submission process.

The original version of the website and abstract submission scripts
were written by Ramzi Mirshak, and were modified and maintained by him
until 2006. Starting with C-DOGS 2007, Clark Richards took over
maintenance and operation of the website from Ramzi, following a
series of tutorials into the workings of the various scripts and
pages.

The reason for the creation of this manual is to make the system more
portable, by elucidating the various components of the abstract
submission system so that in the future it can continue to be
maintained. The current version incorporates shell scripts, html, php,
\LaTeX, and perl, all of which interact to produce the list of
speakers and pdf abstracts that you can see on the
website\footnote{\url{www.phys.ocean.dal.ca/dosa/cdogs/}}.

In addition, changes have been made following C-DOGS 2009 to make the
process easier on the maintainer, including:
\begin{itemize}
\item adding increased clarity and handling for the submission scripts
  (i.e. recognition of strange characters and more warnings about
  cutting and pasting from MSWord documents)
\item allowing the schedule webpages to be generated automatically, and
\item and allowing the proceedings booklet to be generated
  automatically using the determined schedule and the submitted
  abstracts.
\end{itemize}
In the past this step has been very tedious, as it must be done by
cutting and pasting, and the subtle nuances of making the final
version are often forgotten from one year to the next.

The ideal solution to the online submission and booklet/webpage
creation process would be to use a database enable site, for example
SQLite. This would require a significant time commitment to code and
setup, and so I leave this for someone who has more experience in it
than I do.

\section{General Description of Process}

The cdogs website files are located in a subdirectory of the dosa web
directory, hosted on the phys.ocean.dal.ca known as passage. The
directory can be found at: /var/www/html/skye/dosa/. Of course, the
user must have the correct privileges to edit pages (group dosa-web?),
which can be sorted out with the phys.ocean system administrator.

The C-DOGS directories are contained in the dosa web directory, and each
year has been archived. The cdogs link must be updated to point to the
most recent directory so that the webpage functions without an
explicit year.

When it is time to get the website ready for the conference, the first
steps are to update the INFO files, and the welcome.php and list.php
portions of the main site. These procedures are outlined further in
Section \ref{s:initial}.

Following this, once the online submission has opened, the authors
submit abstracts using the online forms described in \ref{s:abs}. The
current incarnation of the website stores the submitted abstracts as a
plain text file with xml-like fields in a hidden directory called
``\verb|$CDOGS/.submissions/|". The maintainer then manually runs the
scripts (located in \verb|$CDOGS/scripts/|) which parse all \verb|.txt| files
in the submissions directory to create \verb|.tex| files for both the
entire abstract on its own, and an abstract part to be used in the
creation of the conference booklet. Another script updates the
speaker list for submitted abstracts on the webpage. A final script
creates a plain text file of all the email addresses of presenters to
facilitate group emailing.

Once all abstracts have been submitted, the schedule can be decided by
the CDOGS committee. The schedule (including breaks and other
scheduled items such as opening and closing remarks) is then entered
into a text file manually, using a specific format that can be parsed
by the necessary schedule script. Running this script creates a
\verb|schedule.php| file for the website. The schedule file is also
used for the creation of the abstract booklet, so that the table of
contents and all of the abstracts are listed in the correct order. 

\section{Initial Steps}\label{s:initial}

\begin{itemize}
\item Copy the template over to passage. This is most easily done by:
  \begin{itemize}
  \item clone the public Git repo at: 
    \url{https://github.com/richardsc/cdogs_template} to \verb|/var/www/html/skye/dosa/|
  \item rename the directory:
\begin{verbatim}
  $ mv cdogs_template cdogsYYYY
\end{verbatim}
    where YYYY is the year
  \item delete the Git history from the clone:
\begin{verbatim}
  $ rm -rf cdogsYYYY/.git
\end{verbatim}
  \item update the symlink:
\begin{verbatim}
  $ rm cdogs
  $ ln -s cdogsYYYY cdogs
\end{verbatim}
  \item change the group permissions to \verb|dosaweb|:
\begin{verbatim}
  $ chgrp -R dosa-web cdogsYYYY
\end{verbatim}
  \end{itemize}
\item change dates (and location where applicable) in all relevant files
  \vspace{-1em}
\begin{verbatim}
  $CDOGS/index.php
  $CDOGS/head.php
  $CDOGS/welcome.php
  $CDOGS/scripts/cdogs_abstract_tex_template_top
  $CDOGS/scripts/cdogs_abstract_tex_template_bottom
  $CDOGS/abstracts/booklet/cdogs_abstract_tex_template_book_top
  $CDOGS/abstracts/booklet/cdogs_abstract_tex_template_book_bottom
  $CDOGS/abstracts/submitted.php
\end{verbatim}
\item Set counter to abstract opening in \verb|submit.html|
\item Make sure relevant logos (and links) are displayed in the
  sidebar (\verb|list.php|)
\item Add a link to the abstract booklet from the previous year, e.g.:
  \begin{itemize}
  \item copy the file
    \verb|dosa/cdogsYYYY/abstract/booklet/cdogsYYYY_abstracts.pdf| to
    \verb|dosa/content/publications/|
  \item create a link to that item following the existing ones already
    in the sidebar
  \end{itemize}
\item CDOGS links in sidebar prior to Abstract submission opening
  should be: 
  \begin{itemize}
  \item Welcome (welcome.php), and
  \item Submit Abstract (submit.html). Note that submit.html doesn't
    go to the proper submit page (submit/submit1.php), but to a
    javascript counter that counts down to the opening of abstract
    submissions.
  \item Sponsors (and logos)
  \item Previous conference booklets
  \end{itemize}


\end{itemize}


\section{Abstract Submission}\label{s:abs}

Steps to open Abstract submission:
\begin{itemize}
\item Update sidebar with proper Submit Abstract link (comment out the
  countdown one and uncomment the one that points to
  \verb|$CDOGS/submit/submit1.php|.
\item Update email address for Apache submission notifications, in
  \verb|$CDOGS/submit/submit6.php|.
\item Verify that the permissions of \verb|$CDOGS/.submissions| are \verb|drwxrwxrwx| or 777.
\item Wait for the submissions to roll in.
\item Periodically login to Passage, and run the abstract scripts in
  \verb|$CDOGS/scripts/| to process the submissions. It can be a good
  idea to try to run them after each submission, in case there are
  problems with the submitted abstract (see next section). Future
  versions will run the script automatically, either based on a cron
  job at regular intervals, or in response to the completed
  submission.
\item The scripts to run are:
\begin{verbatim}
./convert_abstract_to_tex.pl
./update_speakerlist.pl
\end{verbatim}
  The first creates the pdf of the submitted abstract using \LaTeX\
  (located in \verb|$CDOGS/abstracts/|), while the second updates the
  file \verb|$CDOGS/abstracts/speaker_list.html|.
\end{itemize}


\subsection{Common problems with abstract submission}

\begin{itemize}
\item Names with middle initial, or more than just first and last
  name. THIS WILL BE FIXED.
\item Abstracts cut and paste from a non-text editing program, like
  MSWord. This can create several different problems:
  \begin{itemize}
  \item Abstract text with non-latex symbols in it. The most common
    are: percent (e.g. $\%$) without the slash, degrees
    (e.g. $^{\circ}$), accents, italics (species names, etc), units,
    super/subscripts, etc.
  \item Funny placement of carriage return characters (e.g. \verb|^M|)
  \end{itemize}

\end{itemize}

\section{Schedule Creation}

\section{Booklet Creation}

The proceedings booklet is created by combining the various required
templates with the output from the \verb|convert_abstract_to_tex.pl|
script. This is currently a mostly manual process, and involves some
cutting in pasting in the correct order, with manual update of page
numbers at the end. The steps involved will be detailed here.

The four files needed to create the booklet are:
\begin{itemize}
\item \verb|$CDOGS/abstracts/booklet/cdogs_abstract_tex_template_book_top|
\item \verb|$CDOGS/abstracts/booklet/cdogs_abstract_tex_template_book_bottom|
\item \verb|$CDOGS/scripts/booklet_sched_part.tex|
\item \verb|$CDOGS/scripts/booklet_abs_part.tex|
\end{itemize}

Once all the abstracts have all been submitted,
\verb|convert_abstract_to_tex.pl| has been run on everything (without
errors), and the schedule has been determined, the proceedings booklet
can be created.

The manual steps required to prepare the files for the booklet are:
\begin{itemize}
\item copy \verb|$CDOGS/scripts/booklet_sched_part.tex| and
  \verb|$CDOGS/scripts/booklet_abs_part.tex| to
  \verb|$CDOGS/abstracts/booklet/|
\item edit \verb|booklet_sched_part.tex| to fill in the correct
  schedule order and times. Leave the \verb|page| field unchanged for
  now
\item re-order the items in \verb|booklet_abs_part.tex| to match the
  correct order in the schedule.
\item insert \verb|\newpage| commands where appropriate so that an
  abstract doesn't span two pages. This step may require \LaTeX ing
  the document several times to make sure it looks ok.
\item record the page numbers for each abstract, and then edit
  \verb|booklet_sched_part.tex| to enter the appropriate page numbers
  for each talk.
\end{itemize}

The final \LaTeX\ document to create the booklet is then the result of
concatenating the four files listed above, in the following order:
\begin{itemize}
\item \verb|$CDOGS/abstracts/booklet/cdogs_abstract_tex_template_book_top|
\item \verb|$CDOGS/scripts/booklet_sched_part.tex|
\item \verb|$CDOGS/scripts/booklet_abs_part.tex|
\item \verb|$CDOGS/abstracts/booklet/cdogs_abstract_tex_template_book_bottom|
\end{itemize}
and running \LaTeX\ to produce a pdf. If necessary, run \LaTeX\ in
nonstopmode, i.e.:
\begin{verbatim}
$ cat cdogs_abstract_tex_template_book_top booklet_sched_part.tex
booklet_abs_part.tex cdogs_abstract_tex_template_book_bottom >
cdogsYYYY_abstracts.tex
$ pdflatex -interaction=nonstopmode cdogsYYYY_abstracts.tex
\end{verbatim}

\end{document}

