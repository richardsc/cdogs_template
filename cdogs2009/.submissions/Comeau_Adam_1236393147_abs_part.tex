\begin{minipage}{\linewidth}\begin{center}\begin{minipage}{\linewidth}
  \abTitle{Methods: describing phytoplankton physiological variability} \vspace{2 mm} \begin{center}
  \abSpeaker{Adam~J.~Comeau}{1}\abCoauthorO{John Cullen}{1}  \vspace{2 mm}\begin{center}
  
  $\abAffilO{Department of Oceanography, Dalhousie University, Halifax, NS, B3H 4J1, Canada}{1}$

  \end{center}
  \vspace{2 mm}\abEmail{adam.comeau@dal.ca}
  \end{center}\end{minipage}\end{center}
  \begin{center}\rule{0.70\linewidth}{0.5 pt}\end{center}
  \begin{minipage}{\linewidth}
\noindent By understanding factors that influence parameters related to photosynthesis, better estimates of primary productivity and particle dynamics can be obtained. We describe a new method to estimate photoacclimation, a physiological process that influences both photosynthesis vs. irradiance (\sl P \rm vs. \sl E\rm) parameters and chemical composition of phytoplankton, based on profiles of \sl in situ \rm fluorescence and irradiance and apply it to examine the variability of phytoplankton photoacclimation in relation to environmental variables.
Profiles of \sl in situ \rm chlorophyll fluorescence have been routinely measured during oceanographic surveys for several decades. Near surface decreases of fluorescence yield, chlorophyll fluorescence normalized to some measure of phytoplankton biomass, are commonly observed during daytime profiles. This decrease in fluorescence is due to physiological processes, activated in high irradiance, which act to dissipate light energy absorbed by phytoplankton. Lab studies show that the irradiance at which this quenching of fluorescence yield begins is related to a parameter used to estimate primary productivity. With the simple requirements of irradiance and fluorescence yield profiles, this method can be applied to many existing datasets. Examining variations of the light level where fluorescence quenching begins in response to environmental variables such as average light in the mixed layer, will provide new information on how phytoplankton acclimate to their environment.
\end{minipage}\end{minipage}
