\begin{minipage}{\linewidth}\begin{center}\begin{minipage}{\linewidth}
  \abTitle{The zooplankton biomass size-spectrum:  bottom-up vs top-down ecosystem functioning in the Northumberland Strait.} \vspace{2 mm} \begin{center}
  \abSpeaker{Julie~Sperl}{1}\abCoauthorO{Taggart}{1}\abCoauthorO{J.M. Hanson}{2}  \vspace{2 mm}\begin{center}
  
  $\abAffilO{Department of Oceanography, Dalhousie University, Halifax, NS, B3H 4J1, Canada}{1}$

  
  $\abAffilO{ DFO, Ecosystems Research, Oceans & Science Branch, Gulf Fisheries Centre, P. O. Box 5030, Moncton, NB E1C 9B6}{2}$

  \end{center}
  \vspace{2 mm}\abEmail{jl583069@dal.ca}
  \end{center}\end{minipage}\end{center}
  \begin{center}\rule{0.70\linewidth}{0.5 pt}\end{center}
  \begin{minipage}{\linewidth}
\noindent The Northumberland Strait is a flow-through system fed by several nutrient-rich estuaries. The system is characterized by several gaps in information; particularly zooplankton production and fate and its contribution to regional ecosystem functioning, especially from a fisheries perspective. Nutrient enhancement from estuaries should increase primary production in the Strait. Consequently, increased biomass of small zooplankton should increase the slope of the zooplankton biomass size-frequency distribution; i.e. the biomass size spectrum (BSS) wherein the slope is an index of secondary production. With continuing production (bottom-up control), biomass should be transferred via predation to larger particles, thus reducing the slope while increasing the total biomass and the intercept of the BSS. Alternatively, ichthyoplankton feeding on larger particles may increasing the BSS slope via top-down control. The goal of this study is to address the above phenomena by collecting zooplankton diversity (species, abundance at size etc.) and distribution (x, y, z, t) estimates from estuaries and along the Strait. The target BSS will be in the 200μm to 10 mm size-range. Sampling will occur in summer and autumn in each of 2009 and 2010. Samples will be collected in the Strait using a BIONESS net-sampling system fitted with an optical particle counter (OPC) for use in cross-calibrating biological samples with OPC data. Another OPC mounted on a V-fin will be used for rapid, large-scale, tow-yo sampling of the estuaries and the Strait. Data analyses will focus on temporal and spatial variation in the biomass size spectrum and its utility in determining top-down and bottom-up changes in the ecosystem.
\end{minipage}\end{minipage}
