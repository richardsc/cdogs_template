\begin{minipage}{\linewidth}\begin{center}\begin{minipage}{\linewidth}
  \abTitle{Carbon Cycling in the Arctic Archipelago: The Export of Pacific Carbon to the North Atlantic } \vspace{2 mm} \begin{center}
  \abSpeaker{Elizabeth~Shadwick}{1}\abCoauthorO{Tim Papakyriakou}{2}\abCoauthorO{Friederike Prowe}{1,3}\abCoauthorO{Doris Leong}{1}\abCoauthorO{Stephanie Moore}{1}\abCoauthorO{Helmuth Thomas}{1}  \vspace{2 mm}\begin{center}
  
  $\abAffilO{Department of Oceanography, Dalhousie University, Halifax, NS, B3H 4J1, Canada}{1}$

  
  $\abAffilO{Center for Earth Observation Science, University of Manitoba, Winnipeg MB, Canada}{2}$

  
  $\abAffilO{Leibniz-Institut fur Meereswissenschaften, IFM-GEOMAR, D-24105 Kiel, Germany}{3}$

  \end{center}
  \vspace{2 mm}\abEmail{elizabeth.shadwick@dal.ca}
  \end{center}\end{minipage}\end{center}
  \begin{center}\rule{0.70\linewidth}{0.5 pt}\end{center}
  \begin{minipage}{\linewidth}
\noindent The Arctic Ocean is expected to be disproportionately sensitive to climatic changes, and is thought to be an area where such changes might be detected. The Arctic hydrological cycle is influenced by: runoff and precipitation, sea ice formation/melting, and the inflow of saline waters from Bering and Fram Straits and the Barents Sea Shelf.  Pacific water is recognizable as lower salinity water, with high concentrations of dissolved inorganic carbon (DIC), flowing from the Arctic Ocean to the North Atlantic via the Canadian Arctic Archipelago. 
We present DIC data from an east-west section through the Archipelago, as part of the Canadian International Polar Year initiatives. The fractions of Pacific and Arctic Ocean waters leaving the Archipelago and entering Baffin Bay, and subsequently the North Atlantic, are computed. The eastward transport of carbon from the Pacific, via the Arctic, to the North Atlantic is estimated. 

Altered mixing ratios of Pacific and freshwater in the Arctic Ocean have been recorded in recent decades. Any climatically driven alterations in the composition of waters leaving the Arctic Archipelago may have implications for anthropogenic CO$_2$ uptake, and hence ocean acidification, in the sub-polar and temperate North Atlantic.
\end{minipage}\end{minipage}
