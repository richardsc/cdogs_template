\begin{minipage}{\linewidth}\begin{center}\begin{minipage}{\linewidth}
  \abTitle{Reconstructing precipitation variations 0-30 kyrs BP in the western equatorial Pacific using organic biomarkers} \vspace{2 mm} \begin{center}
  \abSpeaker{Katherine~Hastings}{1}\abCoauthorO{Markus Kienast}{1}  \vspace{2 mm}\begin{center}
  
  $\abAffilO{Department of Oceanography, Dalhousie University, Halifax, NS, B3H 4J1, Canada}{1}$

  \end{center}
  \vspace{2 mm}\abEmail{katherine.hastings@dal.ca}
  \end{center}\end{minipage}\end{center}
  \begin{center}\rule{0.70\linewidth}{0.5 pt}\end{center}
  \begin{minipage}{\linewidth}
\noindent The western equatorial Pacific (WEP) is a highly dynamic region, playing a significant role in both tropical and extra-tropical climates through its strong influence on the El Ni\~{n}o-Southern Oscillation (ENSO) phenomenon, the East Asian monsoon, and the position of the Inter-tropical Convergence Zone (ITCZ).  While the modern behaviours of these climate systems are well understood, past behaviours are subject to debate.  We intend to reconstruct paleo-precipitation patterns in the WEP, which are linked to ENSO and ITCZ dynamics, by quantifying n-alkane composition and abundance in samples from several sediment cores along a N-S transect.  N-alkanes are unique to terrestrial leaf waxes, and can thus be used to monitor fluvial influx of terrigenous material into the ocean from adjacent landmasses.  The simplistic assumption is that variations in regional precipitation can be inferred from changes in this influx; however, there are other factors, such as sea level fluctuation, which may affect the sedimentary record and complicate our interpretation.  We hypothesize that zonal shifts in precipitation due to ENSO-like variability should affect all of our core sites similarly.  In contrast, meridional precipitation shifts due to ITCZ migration should result in opposing trends at our core sites.  Here, we present preliminary results from three core sites and discuss what they suggest with respect to resolving glacial-interglacial climate and ocean dynamics.           
\end{minipage}\end{minipage}
