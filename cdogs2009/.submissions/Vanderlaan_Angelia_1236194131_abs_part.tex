\begin{minipage}{\linewidth}\begin{center}\begin{minipage}{\linewidth}
  \abTitle{Ships voluntarily alter course to protect endangered whales} \vspace{2 mm} \begin{center}
  \abSpeaker{Angelia~S.M.~Vanderlaan}{1}\abCoauthorO{Christopher T. Taggart}{1}  \vspace{2 mm}\begin{center}
  
  $\abAffilO{Department of Oceanography, Dalhousie University, Halifax, NS, B3H 4J1, Canada}{1}$

  \end{center}
  \vspace{2 mm}\abEmail{avanderl@phys.ocean.dal.ca}
  \end{center}\end{minipage}\end{center}
  \begin{center}\rule{0.70\linewidth}{0.5 pt}\end{center}
  \begin{minipage}{\linewidth}
\noindent Ocean-going vessels pose a threat to large whales worldwide and are responsible for the majority of deaths diagnosed among endangered North Atlantic right whales (\sl Eubalaena glacialis\rm). Various conservation measures, including vessel re-routing and vessel-speed restrictions, have been implemented to reduce vessel-strike mortality in this species. We initiated the Vessel Avoidance & Conservation Area Transit Experiment (VACATE) to evaluate the efficacy of a voluntary and seasonal Area to be Avoided (ATBA) in reducing the risk of lethal vessel-strikes. The ATBA was adopted by the International Maritime Organization (IMO) for the Roseway Basin region of the Scotian Shelf in 2008. The effectiveness of this vessel-avoidance scheme in reducing risk without the imposition of vessel-speed restrictions is entirely dependent on vessel-operator compliance.  Using a network of Automatic Identification System receivers we collected static, dynamic, and voyage-related vessel data in the Roseway Basin region, both pre- and post-implementation of the ATBA.  Our analyses show that semimonthly estimates of vessel-operator voluntary compliance range from 57\% to 87\%, and stabilised at 71\% within the first 5 months of implementation.  Using pre- and post-implementation vessel-navigation and speed data, along with right whale sightings per unit effort data, we estimate an 82\% reduction in the risk of lethal vessel-strikes to right whales that is due to vessel-operator compliance. The high level of compliance achieved with this voluntary conservation initiative is likely due to the ATBA being adopted by the IMO. Through VACATE we demonstrate that the international shipping industry is able and willing to voluntarily alter course to protect endangered whales.  
\end{minipage}\end{minipage}
