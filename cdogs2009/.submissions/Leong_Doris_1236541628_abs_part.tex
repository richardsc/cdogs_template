\begin{minipage}{\linewidth}\begin{center}\begin{minipage}{\linewidth}
  \abTitle{Does turbulence sound the same from over here, under there, everywhere?} \vspace{2 mm} \begin{center}
  \abSpeaker{Doris~Leong}{1}\abCoauthorO{Tetjana Ross}{1}  \vspace{2 mm}\begin{center}
  
  $\abAffilO{Department of Oceanography, Dalhousie University, Halifax, NS, B3H 4J1, Canada}{1}$

  \end{center}
  \vspace{2 mm}\abEmail{doris.leong@dal.ca}
  \end{center}\end{minipage}\end{center}
  \begin{center}\rule{0.70\linewidth}{0.5 pt}\end{center}
  \begin{minipage}{\linewidth}
\noindent Small-scale ocean turbulence can manifest as temperature and salinity microstructure that scatter sound. Acoustic measurements of turbulent fluctuations are translated into physical parameters using models that generally assume isotropic scattering, although there is no clear experimental evidence to support this. We investigate the existence of anisotropy in small-scale turbulence generated by internal waves. Strong anisotropy is predicted to noticeably shift the dissipative roll-off in the spectral frequency response of turbulence to lower frequencies. Observations of scatter from turbulence are made using a broadband acoustic system that is capable of sampling vertically or horizontally through the water column. The overall scatter shows statistical patterns in spectral shape that suggest the presence of anisotropy in either biological or physical scatter. Turbulence dissipation rates are estimated from acoustic inversions of spectra and yield no clear evidence of small-scale turbulence anisotropy.
\end{minipage}\end{minipage}
