\begin{minipage}{\linewidth}\begin{center}\begin{minipage}{\linewidth}
  \abTitle{Development of a nested-grid shelf circulation model using OPA for the eastern Canadian shelf} \vspace{2 mm} \begin{center}
  \abSpeaker{Jorge~R.~Urrego-Blanco}{1}\abCoauthorO{Jinyu Sheng}{1}  \vspace{2 mm}\begin{center}
  
  $\abAffilO{Department of Oceanography, Dalhousie University, Halifax, NS, B3H 4J1, Canada}{1}$

  \end{center}
  \vspace{2 mm}\abEmail{jorge.urrego.blanco@dal.ca}
  \end{center}\end{minipage}\end{center}
  \begin{center}\rule{0.70\linewidth}{0.5 pt}\end{center}
  \begin{minipage}{\linewidth}
\noindent As a first step of developing a nested-grid circulation model for the eastern Canadian shelf, we constructed a coarse-grid (1/4$^\circ$) northwest Atlantic circulation model using the ocean general circulation model known as OPA (Oc\'{e}an PArall\'{e}lis\'{e}). The model domain covers the area between 32$^\circ$W and 81$^\circ$W and between 33$^\circ$N and 57$^\circ$N. This model was used to simulate the 3-D circulation from 1990 to 1999 in this study. The model was forced by atmospheric reanalysis fields produced by Large and Yeager (2004) and monthly mean climatologies of temperature and salinity produced by Geshelin et al. (1999). Three different numerical experiments were conducted to examine the model performance in simulating large-scale circulation over the study region. These three experiments are: a) a fully prognostic run without data assimilation; b) a run using the spectral nudging method (Thompson et al. 2006); and c) a run using the semi-prognostic method (Sheng et al. 2001). In the first experiment no hydrographical assimilation is made and model results in this experiment demonstrate significant model drift and unrealistic circulation features for a multi-year model integration. For the spectral nudging experiment the model drift in TS fields is significantly reduced and the hydrographical seasonal cycle is well reproduced by the model as expected. However, tracer variability in this run is strongly damped and the eddy field is less free to evolve in the model. In the semi-prognostic run a correction (or assimilation) term is introduced in the model through the hydrostatic equation while leaving the model tracer equations to be fully prognostic. Model results demonstrate that the semi-prognostic method not only reduces model drift but also improves the flow field simulation. This method however still damps the mesoscale eddy field. Future work will include the use of a smoothed semi-prognostic method which allows the mesoscale eddy field to be more realistically reproduced.
\end{minipage}\end{minipage}
