\begin{minipage}{\linewidth}\begin{center}\begin{minipage}{\linewidth}
  \abTitle{Long Wavelength Ripples in the Nearshore} \vspace{2 mm} \begin{center}
  \abSpeaker{Trajce~Alcinov}{1}\abCoauthorO{Alex Hay}{1}  \vspace{2 mm}\begin{center}
  
  $\abAffilO{Department of Oceanography, Dalhousie University, Halifax, NS, B3H 4J1, Canada}{1}$

  \end{center}
  \vspace{2 mm}\abEmail{t.alcinov@dal.ca}
  \end{center}\end{minipage}\end{center}
  \begin{center}\rule{0.70\linewidth}{0.5 pt}\end{center}
  \begin{minipage}{\linewidth}
\noindent Sediment bedforms are ubiquitous in the nearshore environment, and their characteristics and evolution have a direct effect on the hydrodynamics and the rate of sediment transport. The focus of this study is long wavelength ripples (LWRs) observed at two locations in the nearshore at roughly 3m water depth under combined current and wave conditions in Duck, North Carolina. The observed LWRs are straight-crested bedforms with wavelengths in the range of 30-75 cm. They occur during the build up of storms, when the incident wave direction is rapidly changing, possibly due to the migration of the center of a storm. A principal goal of the study is to test the maximum gross bedform-normal transport (mGBNT) hypothesis, which states that the orientation of ripples in directionally varying flows is such that the gross sediment transport normal to the ripple crest is maximized. Ripple wavelengths and orientation are measured from rotary fanbeam images and current and wave conditions are obtained from electromagnetic (EM) flowmeters and an offshore pressure gauge array. Tests of the mGBNT hypothesis in which the transport was calculated using a sediment transport model indicate that it is not a good predictor of LWR orientation. The observed LWR orientation seems to be tied to the incident wave direction, with an additional offset the sign of which depends on the sign of the longshore current.
\end{minipage}\end{minipage}
