\begin{minipage}{\linewidth}\begin{center}\begin{minipage}{\linewidth}
  \abTitle{Influence of density-dependent food consumption, foraging and stacking behaviour on the growth rate of the Northern abalone, Haliotis kamtschatkana} \vspace{2 mm} \begin{center}
  \abSpeaker{Michelle~Lloyd}{1}\abCoauthorO{Amanda Bates}{1}  \vspace{2 mm}\begin{center}
  
  $\abAffilO{Bamfield Marine Sciences Centre, Bamfield, BC, V0R 1B0, Canada}{1}$

  \end{center}
  \vspace{2 mm}\abEmail{michelle.lloyd@dal.ca}
  \end{center}\end{minipage}\end{center}
  \begin{center}\rule{0.70\linewidth}{0.5 pt}\end{center}
  \begin{minipage}{\linewidth}
\noindent Growth of abalone in the wild and hatchery is density-dependent in response to intraspecific competition for food and/or space. To determine if a candidate aquaculture species, Haliotis kamtschatkana, exhibits density-dependent growth we raised animals at three density levels and two food treatments: unlimited (ad libitum) and rationed (individual portions were the same among density treatments). We also tested for differences in food consumption, foraging patterns and stacking behaviour among the density levels. We observed density-dependent growth in the rationed treatments, indicating that relatively high growth rates at lower densities are driven, in part, by factors other than differences in food consumption.However, overall the quantity of food consumed related directly to growth; treatments fed ad libitum had higher growth rates. Furthermore, even when food was provided in excess, foraging was restricted to $\sim$2 h after sunset in all treatments and the amount consumed per abalone was significantly lower at high densities. This is probably because high density animals could not access the food provided: fewer were observed foraging and they had to move from prominent stacks. Our results indicate that both temporal and spatial access to food are critical and that managers can observe foraging and stacking by abalone in tanks to determine if a specific design will limit food consumption, and ultimately growth.
\end{minipage}\end{minipage}
