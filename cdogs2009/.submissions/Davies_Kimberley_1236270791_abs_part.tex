\begin{minipage}{\linewidth}\begin{center}\begin{minipage}{\linewidth}
  \abTitle{Physical forcing of space-time variation in the copepod prey field of North Atlantic right whales } \vspace{2 mm} \begin{center}
  \abSpeaker{Kimberley~T.~A.~Davies}{1}\abCoauthorO{Christopher T. Taggart}{1}  \vspace{2 mm}\begin{center}
  
  $\abAffilO{Department of Oceanography, Dalhousie University, Halifax, NS, B3H 4J1, Canada}{1}$

  \end{center}
  \vspace{2 mm}\abEmail{kim.davies@dal.ca}
  \end{center}\end{minipage}\end{center}
  \begin{center}\rule{0.70\linewidth}{0.5 pt}\end{center}
  \begin{minipage}{\linewidth}
\noindent Defining habitat critical for the survival of endangered species is the goal of many marine science initiatives. In the pelagic zone, feeding habitat boundaries are difficult to define statically in space because the prey-field is subject to advection and mixing by regional flow fields. Here we address this issue in relation to defining the critical feeding habitat of the endangered North Atlantic right whale. These whales feed on diapausing, lipid rich copepods that are aggregated in high concentrations, near bottom, in the deep (>100m) basins of the Scotia-Fundy region. We ask can variation in the spatial distribution of the right whale prey-field at short (tidal, diel) time scales be explained by variation in the current regime in Roseway Basin. Current speed and acoustic backscatter (zooplankton abundance) data were simultaneously collected using moorings at three locations in the Basin. Two moorings, fitted with upward looking Acoustic Doppler Current Profilers (ADCP; one 300 kHz and one 600 kHz), downward looking Aquadopp profilers, and CTDs (SBE-37), straddled a sloping region on the edge of the Basin where whales feed and were located within one tidal excursion from one another. The third mooring (upward looking 300 kHz ADCP) was located at the deepest portion of the Basin. The area was concurrently surveyed using a ship-mounted echo-sounder (Simrad, 120 kHz). We illustrate how zooplankton abundance varies in response to tidal advection of water masses and the residual circulation. 
\end{minipage}\end{minipage}
