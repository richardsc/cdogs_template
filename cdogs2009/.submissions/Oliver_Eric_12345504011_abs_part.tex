\begin{minipage}{\linewidth}\begin{center}\begin{minipage}{\linewidth}
  \abTitle{The Madden-Julian Oscillation and Local and Remote Forcing of the Ocean} \vspace{2 mm} \begin{center}
  \abSpeaker{Eric~Oliver}{1}\abCoauthorO{Keith Thompson}{1}  \vspace{2 mm}\begin{center}
  
  $\abAffilO{Department of Oceanography, Dalhousie University, Halifax, NS, B3H 4J1, Canada}{1}$

  \end{center}
  \vspace{2 mm}\abEmail{eric.oliver@phys.ocean.dal.ca}
  \end{center}\end{minipage}\end{center}
  \begin{center}\rule{0.70\linewidth}{0.5 pt}\end{center}
  \begin{minipage}{\linewidth}
\noindent The Madden-Julian Oscillation (MJO) is the dominant mode of atmospheric variability in the tropical atmosphere on intraseasonal timescales (i.e., weeks to seasons).  It is an eastward-propagating phenomenon with clear expressions in outgoing longwave radiation, precipitation and zonal wind stress over the tropical oceans.  The MJO has the potential to help bridge the gap between between extended-range weather forecasts and seasonal climate forecasts of both the atmosphere and ocean.  Observational and modeling studies have shown that the MJO can drive variability in the tropical ocean through local heat and momentum fluxes. In this study we examine the connection between sea level and the MJO on a global scale. We first identify regions exhibiting a significant (both statistical and practical) relationship between sea level and the MJO. The first region consists of the equatorial Pacific and western coast zones of North and South America. Consistent with previous studies, we identify wind-driven equatorially trapped Kelvin waves propagating eastward along the equatorial Pacific and then poleward along the coastal trapped waveguides of North and South America.  The second region includes the shallow waters of the Gulf of Carpentaria along the north coast of Australia and the adjacent Arafura and Timor Seas. Sea level set up by onshore winds is shown to be the dominant physical process. Finally, the northeastern Indian Ocean is shown to be a complex region involving a combination of equatorially trapped Kelvin waves, coastal trapped waves and westward propagating Rossby waves exhibiting characteristics of both local and remote forcing. The implications for deep and coastal ocean forecasting are discussed.
\end{minipage}\end{minipage}
