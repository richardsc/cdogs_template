\begin{minipage}{\linewidth}\begin{center}\begin{minipage}{\linewidth}
  \abTitle{Sea level rise: A better understanding from new satellite measurements} \vspace{2 mm} \begin{center}
  \abSpeaker{Simon~Higginson}{1}  \vspace{2 mm}\begin{center}
  
  $\abAffilO{Department of Oceanography, Dalhousie University, Halifax, NS, B3H 4J1, Canada}{1}$

  \end{center}
  \vspace{2 mm}\abEmail{simon.higginson@dal.ca}
  \end{center}\end{minipage}\end{center}
  \begin{center}\rule{0.70\linewidth}{0.5 pt}\end{center}
  \begin{minipage}{\linewidth}
\noindent The geoid represents global mean sea level if the oceans were at rest. Mean dynamic topography (MDT) is the mean variation of the height of the ocean relative to the geoid, and can be related to the mean circulation by an assumption of geostrophy. Poor knowledge of the geoid has prevented direct measurement of MDT, and estimates have been based on indirect methods using hydrographic data. However the GRACE satellite gravity mission is providing detailed information on the geoid and its temporal change, leading to improved estimates of MDT. GRACE also provides the mass change of the ocean which, when combined with hydrographic data, provides valuable information on the relative contribution of ice melt and ocean warming to total sea level rise. Studies of the spatial pattern of sea level change have tended to simplify the ocean response. There is scope to introduce more realism, considering future changes to MDT and the circulation resulting from the predicted total sea level rise.
\end{minipage}\end{minipage}
