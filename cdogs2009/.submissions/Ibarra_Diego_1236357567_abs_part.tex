\begin{minipage}{\linewidth}\begin{center}\begin{minipage}{\linewidth}
  \abTitle{A 3-D physical-biological model to assess the effect of mussel aquaculture on water-column dynamics in Ship Harbour, Nova Scotia} \vspace{2 mm} \begin{center}
  \abSpeaker{Diego~A.~Ibarra}{1}\abCoauthorO{Katja Fennel}{1}\abCoauthorO{John J. Cullen}{1}  \vspace{2 mm}\begin{center}
  
  $\abAffilO{Department of Oceanography, Dalhousie University, Halifax, NS, B3H 4J1, Canada}{1}$

  \end{center}
  \vspace{2 mm}\abEmail{dibarra@dal.ca}
  \end{center}\end{minipage}\end{center}
  \begin{center}\rule{0.70\linewidth}{0.5 pt}\end{center}
  \begin{minipage}{\linewidth}
\noindent We examined the water-column impacts of mussel farming in Ship Harbour (Nova Scotia) using the Regional Ocean Modeling System (ROMS) coupled with an ecosystem model containing a sessile filter-feeder sub-model. For model tuning and ground-truthing, we used data from a variety of bio-optical instruments and water samples, collected during multiple transects conducted at each tidal cycle over 4 days and nights. We used our model to quantify the effect of mussels on water-column variables by estimating the difference between model simulations with and without mussels. The resulting 3-D maps of mussel-associated impacts showed a time-averaged decrease in phytoplankton and small detritus (up to 45 and 15\%, respectively), and an increase in large detritus, ammonia and nitrate (up to 14, 3 and 2\%, respectively). In this work, we demonstrate the applicability of 3-D coupled models for aquaculture management. However, we also emphasize the need for continuous records of at least two independent estimates of phytoplankton to tune and ground-truth models, and ultimately, to understand the impact of bivalve aquaculture on pelagic ecosystems.
\end{minipage}\end{minipage}
