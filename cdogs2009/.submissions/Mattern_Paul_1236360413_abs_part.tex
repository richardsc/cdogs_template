\begin{minipage}{\linewidth}\begin{center}\begin{minipage}{\linewidth}
  \abTitle{Comparing satellite data and model output using image distance measures} \vspace{2 mm} \begin{center}
  \abSpeaker{Paul~Mattern}{1,2}\abCoauthorO{Katja Fennel}{1}\abCoauthorO{Mike Dowd}{2}  \vspace{2 mm}\begin{center}
  
  $\abAffilO{Department of Oceanography, Dalhousie University, Halifax, NS, B3H 4J1, Canada}{1}$

  
  $\abAffilO{Department of Mathematics and Statistics, Dalhousie University, Halifax, NS, B3H 4J1, Canada}{2}$

  \end{center}
  \vspace{2 mm}\abEmail{paul.mattern@Dal.Ca}
  \end{center}\end{minipage}\end{center}
  \begin{center}\rule{0.70\linewidth}{0.5 pt}\end{center}
  \begin{minipage}{\linewidth}
\noindent Quantitative comparison of model results with measured data is an essential part of model skill assessment and data assimilation. Specifically, we are seeking a suitable measure of fit for comparing surface ocean satellite images with corresponding model output. We evaluated a variety of distance measures including the commonly used Root-Mean-Squared (RMS) error, and other metrics from the image comparison literature. In our assessment simple pixel-by-pixel comparison like the RMS error yield unsatisfactory results in many cases. We will present examples that demonstrate the advantages of alternative image distance and fit measures, for example a modified version of the Hausdorff distance, which we adapted for use with (partially incomplete) satellite images.
\end{minipage}\end{minipage}
