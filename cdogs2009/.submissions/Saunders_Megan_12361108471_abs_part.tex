\begin{minipage}{\linewidth}\begin{center}\begin{minipage}{\linewidth}
  \abTitle{Temperature, timing and growth: implications for outbreaks of an introduced species} \vspace{2 mm} \begin{center}
  \abSpeaker{Megan~Saunders}{1}\abCoauthorO{Anna Metaxas}{1}\abCoauthorO{Ramón Filgueira}{1}  \vspace{2 mm}\begin{center}
  
  $\abAffilO{Department of Oceanography, Dalhousie University, Halifax, NS, B3H 4J1, Canada}{1}$

  \end{center}
  \vspace{2 mm}\abEmail{msaunders@dal.ca}
  \end{center}\end{minipage}\end{center}
  \begin{center}\rule{0.70\linewidth}{0.5 pt}\end{center}
  \begin{minipage}{\linewidth}
\noindent Outbreaks of the invasive bryozoan \sl Membranipora membranacea \rm in the western Atlantic facilitate the invasion of other algae by defoliating kelp beds. To examine the effect of temperature on the \sl M. membranacea \rm population, we constructed an individual-based population model, which successfully simulated the timing of onset of settlement, number of adult colonies, maximum colony diameter, and relative interannual patterns in abundance. We used the model to examine the relative effect on the population of varying temperature by -2 to +2$^{\circ}$C day$^{-1}$. Increasing daily temperature by 2$^{\circ}$C caused the population to occur 1 month earlier in the season, and resulted in a 100 fold increase in abundance. Changes in winter and summer temperature had the most pronounced effects on the timing and abundance of the population, respectively. Our results suggest that outbreaks of this species will be more pronounced if temperature increases as a result of climate change. 
\end{minipage}\end{minipage}
