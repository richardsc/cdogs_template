\begin{minipage}{\linewidth}\begin{center}\begin{minipage}{\linewidth}
  \abTitle{Sustainability of scallop populations on Georges Bank: implications of spawning seasonality.} \vspace{2 mm} \begin{center}
  \abSpeaker{Chad~Gilbert}{1}\abCoauthorO{Wendy Gentleman}{2,1}\abCoauthorO{Claudio DiBacco}{3}\abCoauthorO{Catherine Johnson}{3}\abCoauthorO{Jamie Pringle}{4}\abCoauthorO{Changsheng Chen}{5}  \vspace{2 mm}\begin{center}
  
  $\abAffilO{Department of Engineering Mathematics and Internetworking, Dalhousie University, Halifax NS, B3H 4J1, Canada}{1}$

  
  $\abAffilO{Department of Oceanography, Dalhousie University, Halifax, NS, B3H 4J1, Canada}{2}$

  
  $\abAffilO{Bedford Institute of Oceanography, Dartmouth, NS, B2Y 4A2, Canada}{3}$

  
  $\abAffilO{Institue for the study of Earth, Ocean and Space, University of New Hampshire, Durham NH, USA}{4}$

  
  $\abAffilO{Department of Fisheries Oceanography, University of Massachusetts-Dartmouth, New Bedford, MA, USA}{5}$

  \end{center}
  \vspace{2 mm}\abEmail{chad.gilbert@dal.ca}
  \end{center}\end{minipage}\end{center}
  \begin{center}\rule{0.70\linewidth}{0.5 pt}\end{center}
  \begin{minipage}{\linewidth}
\noindent Sea scallops (Placopecten magellanicus) on Georges Bank are important both ecologically and as commercial fisheries. The population is comprised of 3 distinct scallop beds, which are connected via transport of planktonic larvae spawned in the spring and fall. In order to develop sustainable management strategies and predict effects of climate change on the population, we need to quantify how the different beds and spawning times contribute to larval recruitment. 

Here, we calculate larval drift and retention using a 3D particle-tracking model, which couples seasonal currents, larval swimming, turbulent dispersion and larval development. Bed connectivity is quantified, and patterns of larval exchange are shown to differ for each season. Sensitivity to variation in adult distribution, temperature-dependent growth, reproduction and mortality is assessed. Factors controlling long-term success of the scallop population are analyzed using a modified Markov-chain approach. Implications for management of this population in the context of climate change are discussed.
\end{minipage}\end{minipage}

%-----------------------------------------------------------------
\vspace{5 mm} \begin{center}\rule{0.9\linewidth}{1pt}\end{center}
%-----------------------------------------------------------------
\begin{minipage}{\linewidth}\begin{center}\begin{minipage}{\linewidth}
  \abTitle{A 3-D physical-biological model to assess the effect of mussel aquaculture on water-column dynamics in Ship Harbour, Nova Scotia} \vspace{2 mm} \begin{center}
  \abSpeaker{Diego~A.~Ibarra}{1}\abCoauthorO{Katja Fennel}{1}\abCoauthorO{John J. Cullen}{1}  \vspace{2 mm}\begin{center}
  
  $\abAffilO{Department of Oceanography, Dalhousie University, Halifax, NS, B3H 4J1, Canada}{1}$

  \end{center}
  \vspace{2 mm}\abEmail{dibarra@dal.ca}
  \end{center}\end{minipage}\end{center}
  \begin{center}\rule{0.70\linewidth}{0.5 pt}\end{center}
  \begin{minipage}{\linewidth}
\noindent We examined the water-column impacts of mussel farming in Ship Harbour (Nova Scotia) using the Regional Ocean Modeling System (ROMS) coupled with an ecosystem model containing a sessile filter-feeder sub-model. For model tuning and ground-truthing, we used data from a variety of bio-optical instruments and water samples, collected during multiple transects conducted at each tidal cycle over 4 days and nights. We used our model to quantify the effect of mussels on water-column variables by estimating the difference between model simulations with and without mussels. The resulting 3-D maps of mussel-associated impacts showed a time-averaged decrease in phytoplankton and small detritus (up to 45 and 15\%, respectively), and an increase in large detritus, ammonia and nitrate (up to 14, 3 and 2\%, respectively). In this work, we demonstrate the applicability of 3-D coupled models for aquaculture management. However, we also emphasize the need for continuous records of at least two independent estimates of phytoplankton to tune and ground-truth models, and ultimately, to understand the impact of bivalve aquaculture on pelagic ecosystems.
\end{minipage}\end{minipage}

%-----------------------------------------------------------------
\vspace{5 mm} \begin{center}\rule{0.9\linewidth}{1pt}\end{center}
%-----------------------------------------------------------------
\begin{minipage}{\linewidth}\begin{center}\begin{minipage}{\linewidth}
  \abTitle{Reconstructing precipitation variations 0-30 kyrs BP in the western equatorial Pacific using organic biomarkers} \vspace{2 mm} \begin{center}
  \abSpeaker{Katherine~Hastings}{1}\abCoauthorO{Markus Kienast}{1}  \vspace{2 mm}\begin{center}
  
  $\abAffilO{Department of Oceanography, Dalhousie University, Halifax, NS, B3H 4J1, Canada}{1}$

  \end{center}
  \vspace{2 mm}\abEmail{katherine.hastings@dal.ca}
  \end{center}\end{minipage}\end{center}
  \begin{center}\rule{0.70\linewidth}{0.5 pt}\end{center}
  \begin{minipage}{\linewidth}
\noindent The western equatorial Pacific (WEP) is a highly dynamic region, playing a significant role in both tropical and extra-tropical climates through its strong influence on the El Ni\~{n}o-Southern Oscillation (ENSO) phenomenon, the East Asian monsoon, and the position of the Inter-tropical Convergence Zone (ITCZ).  While the modern behaviours of these climate systems are well understood, past behaviours are subject to debate.  We intend to reconstruct paleo-precipitation patterns in the WEP, which are linked to ENSO and ITCZ dynamics, by quantifying n-alkane composition and abundance in samples from several sediment cores along a N-S transect.  N-alkanes are unique to terrestrial leaf waxes, and can thus be used to monitor fluvial influx of terrigenous material into the ocean from adjacent landmasses.  The simplistic assumption is that variations in regional precipitation can be inferred from changes in this influx; however, there are other factors, such as sea level fluctuation, which may affect the sedimentary record and complicate our interpretation.  We hypothesize that zonal shifts in precipitation due to ENSO-like variability should affect all of our core sites similarly.  In contrast, meridional precipitation shifts due to ITCZ migration should result in opposing trends at our core sites.  Here, we present preliminary results from three core sites and discuss what they suggest with respect to resolving glacial-interglacial climate and ocean dynamics.           
\end{minipage}\end{minipage}

%-----------------------------------------------------------------
\vspace{5 mm} \begin{center}\rule{0.9\linewidth}{1pt}\end{center}
%-----------------------------------------------------------------
\begin{minipage}{\linewidth}\begin{center}\begin{minipage}{\linewidth}
  \abTitle{Comparing satellite data and model output using image distance measures} \vspace{2 mm} \begin{center}
  \abSpeaker{Paul~Mattern}{1,2}\abCoauthorO{Katja Fennel}{1}\abCoauthorO{Mike Dowd}{2}  \vspace{2 mm}\begin{center}
  
  $\abAffilO{Department of Oceanography, Dalhousie University, Halifax, NS, B3H 4J1, Canada}{1}$

  
  $\abAffilO{Department of Mathematics and Statistics, Dalhousie University, Halifax, NS, B3H 4J1, Canada}{2}$

  \end{center}
  \vspace{2 mm}\abEmail{paul.mattern@Dal.Ca}
  \end{center}\end{minipage}\end{center}
  \begin{center}\rule{0.70\linewidth}{0.5 pt}\end{center}
  \begin{minipage}{\linewidth}
\noindent Quantitative comparison of model results with measured data is an essential part of model skill assessment and data assimilation. Specifically, we are seeking a suitable measure of fit for comparing surface ocean satellite images with corresponding model output. We evaluated a variety of distance measures including the commonly used Root-Mean-Squared (RMS) error, and other metrics from the image comparison literature. In our assessment simple pixel-by-pixel comparison like the RMS error yield unsatisfactory results in many cases. We will present examples that demonstrate the advantages of alternative image distance and fit measures, for example a modified version of the Hausdorff distance, which we adapted for use with (partially incomplete) satellite images.
\end{minipage}\end{minipage}

%-----------------------------------------------------------------
\vspace{5 mm} \begin{center}\rule{0.9\linewidth}{1pt}\end{center}
%-----------------------------------------------------------------
\begin{minipage}{\linewidth}\begin{center}\begin{minipage}{\linewidth}
  \abTitle{A box model of carrying-capacity for mussel aquaculture in a Norwegian fjord} \vspace{2 mm} \begin{center}
  \abSpeaker{Ramon~Filgueira}{1}\abCoauthorO{Jon Grant}{1}  \vspace{2 mm}\begin{center}
  
  $\abAffilO{Department of Oceanography, Dalhousie University, Halifax, NS, B3H 4J1, Canada}{1}$

  \end{center}
  \vspace{2 mm}\abEmail{ramonf@dal.ca}
  \end{center}\end{minipage}\end{center}
  \begin{center}\rule{0.70\linewidth}{0.5 pt}\end{center}
  \begin{minipage}{\linewidth}
\noindent Shellfish carrying-capacity is determined by the interaction of cultured species with the ecosystem, principally constrained by environmental characteristics and particularly food availability. A recent experiment carried out in Lysefjord (SW Norway) has shown that artificial upwelling of nutrient-rich deeper water stimulated phytoplankton growth, potentially increasing the carrying-capacity for mussel cultivation. With the aim of evaluating aquaculture effects and assisting in development of sustainable mussel culture in Lysefjord, an object-oriented model of environmental-mussel aquaculture interactions and mussel carrying-capacity was constructed. A multiple box ecosystem model was developed with highly configurable GUI-based software (Simile) that allows explicit coupling between boxes, which represent regions of the fjord. Once the box model was developed and calibrated, subsequent application of PEST (Parameter ESTimation) allowed optimization of different variables of the model in order to manage mussel production according to carrying-capacity criteria. The Simile model and the simultaneous application of PEST allowed several scenarios taking into account different stocking densities and the creation of new cultivation areas.
\end{minipage}\end{minipage}

%-----------------------------------------------------------------
\vspace{5 mm} \begin{center}\rule{0.9\linewidth}{1pt}\end{center}
%-----------------------------------------------------------------
\begin{minipage}{\linewidth}\begin{center}\begin{minipage}{\linewidth}
  \abTitle{Physiological and Ecological Interactions of Native Cancer spp. and Carcinus Maenas in British Columbia, Canada} \vspace{2 mm} \begin{center}
  \abSpeaker{Remi~Daigle}{1}\abCoauthorO{Claudio DiBacco}{2}\abCoauthorO{Monica Bravo}{2,3}\abCoauthorO{Tom Therriault}{4}\abCoauthorO{Colin Brauner}{3}\abCoauthorO{Graham Gillespie}{4}  \vspace{2 mm}\begin{center}
  
  $\abAffilO{Department of Oceanography, Dalhousie University, Halifax, NS, B3H 4J1, Canada}{1}$

  
  $\abAffilO{Bedford Institute of Oceanography, Dartmouth, NS, B2Y 4A2, Canada}{2}$

  
  $\abAffilO{University of British Columbia, Vancouver, Canada}{3}$

  
  $\abAffilO{Fisheries and Oceans, Nanaimo, Canada}{4}$

  \end{center}
  \vspace{2 mm}\abEmail{daigleremi@gmail.com}
  \end{center}\end{minipage}\end{center}
  \begin{center}\rule{0.70\linewidth}{0.5 pt}\end{center}
  \begin{minipage}{\linewidth}
\noindent This study characterizes ecological interactions between populations of two native Cancer spp. and the invasive green crab, Carcinus maenas, in Barkley Sound, British Columbia. Laboratory salinity tolerance tests demonstrated that C. maenas had a higher mean tolerance to osmotic stress when compared to Cancer productus and C. gracilis with the latter being the least tolerant. Trapping surveys revealed depth segregated populations of C. gracilis and C. maenas - depth distributions for both spp. fluctuated over time and were significantly related to salinity and ultimately regional rainfall. It is suggested that osmoconforming C. gracilis retreats to deeper waters at times corresponding to a depressed halocline coinciding with heavy freshwater input. Co-occurrence of C. maenas and C. productus was extremely rare. The smaller sized osmoregulating C. maenas seem relegated to areas of high freshwater discharge due to biotic resistance by larger native crabs. These findings suggest that halotolerance may have facilitated the establishment of green crab populations on Canada's west coast. Salinity tolerance data are valuable for assessing the risk of further invasions in estuaries along British Columbia's coast and similar environments.
\end{minipage}\end{minipage}

%-----------------------------------------------------------------
\vspace{5 mm} \begin{center}\rule{0.9\linewidth}{1pt}\end{center}
%-----------------------------------------------------------------
\begin{minipage}{\linewidth}\begin{center}\begin{minipage}{\linewidth}
  \abTitle{Application of variational data assimilation to coupled physical-biological models of the North Atlantic Bloom} \vspace{2 mm} \begin{center}
  \abSpeaker{Witold~Bagniewski}{1,2}\abCoauthorO{Katja Fennel}{1}\abCoauthorO{Mary Jane Perry}{2}\abCoauthorO{Eric D'Asaro}{3}  \vspace{2 mm}\begin{center}
  
  $\abAffilO{Department of Oceanography, Dalhousie University, Halifax, NS, B3H 4J1, Canada}{1}$

  
  $\abAffilO{School of Marine Sciences, University of Maine, Orono ME}{2}$

  
  $\abAffilO{Applied Physics Laboratory, University of Washington, Seattle WA}{3}$

  \end{center}
  \vspace{2 mm}\abEmail{witold.bagniewski@dal.ca}
  \end{center}\end{minipage}\end{center}
  \begin{center}\rule{0.70\linewidth}{0.5 pt}\end{center}
  \begin{minipage}{\linewidth}
\noindent Lagrangian floats and seagliders were deployed in the North Atlantic region south of Iceland from late March to early July 2008 and provided 3-D coverage of the spring bloom over time. The measured physical, chemical and bio-optical data, calibrated with data collected on three supporting cruises, was used to develop an ecosystem model describing the North Atlantic Spring Bloom. The model’s physical framework is based on the 1-D General Ocean Turbulence Model (GOTM) which is set up for a North Atlantic site at 60˚ N, 20˚ W and forced with data on wind speed, air pressure, air temperature and humidity. This physical model is coupled to a biological model that includes small phytoplankton, diatoms, zooplankton, detrital nitrogen, detrital silicate, dissolved inorganic nitrogen, silicic acid, chlorophyll and oxygen. We determined the biological parameters that are most important for model behavior through a sensitivity analysis and will apply variational data assimilation to optimize these. We will present model-based estimates of primary productivity, carbon fluxes and carbon export associated with the bloom.
\end{minipage}\end{minipage}

%-----------------------------------------------------------------
\vspace{5 mm} \begin{center}\rule{0.9\linewidth}{1pt}\end{center}
%-----------------------------------------------------------------
\begin{minipage}{\linewidth}\begin{center}\begin{minipage}{\linewidth}
  \abTitle{A model for bubble rise in soft sediments} \vspace{2 mm} \begin{center}
  \abSpeaker{Christopher~Algar}{1}\abCoauthorO{Dr. Bernard Boudreau}{1}  \vspace{2 mm}\begin{center}
  
  $\abAffilO{Department of Oceanography, Dalhousie University, Halifax, NS, B3H 4J1, Canada}{1}$

  \end{center}
  \vspace{2 mm}\abEmail{calgar@dal.ca}
  \end{center}\end{minipage}\end{center}
  \begin{center}\rule{0.70\linewidth}{0.5 pt}\end{center}
  \begin{minipage}{\linewidth}
\noindent Methane, an important greenhouse gas, is produced in both wetlands and aquatic sediments (marine and freshwater) by the degradation of organic matter under anoxic conditions.  Once produced, methane can migrate from the sediments to the overlying water and eventually the atmosphere by diffusion or bubble ebullition.   Ebullition is a significant source because it can release methane directly to the water column or atmosphere, bypassing the methane oxidizing zone, which consumes much of the diffusive flux of methane.  Here I present a mechanistic model for bubble rise in soft sediments.   The model describes the movement of a single isolated bubble.  The bubble migrates by propagating a fracture and the rate of rise is controlled by the viscoelastic response of the sediments to stresses induced by the bubble.  The model predicts rise velocities as a function of measurable sediment properties and shows that such velocities are significantly faster than the time scale of diffusive release.
\end{minipage}\end{minipage}

%-----------------------------------------------------------------
\vspace{5 mm} \begin{center}\rule{0.9\linewidth}{1pt}\end{center}
%-----------------------------------------------------------------
\begin{minipage}{\linewidth}\begin{center}\begin{minipage}{\linewidth}
  \abTitle{Long Wavelength Ripples in the Nearshore} \vspace{2 mm} \begin{center}
  \abSpeaker{Trajce~Alcinov}{1}\abCoauthorO{Alex Hay}{1}  \vspace{2 mm}\begin{center}
  
  $\abAffilO{Department of Oceanography, Dalhousie University, Halifax, NS, B3H 4J1, Canada}{1}$

  \end{center}
  \vspace{2 mm}\abEmail{t.alcinov@dal.ca}
  \end{center}\end{minipage}\end{center}
  \begin{center}\rule{0.70\linewidth}{0.5 pt}\end{center}
  \begin{minipage}{\linewidth}
\noindent Sediment bedforms are ubiquitous in the nearshore environment, and their characteristics and evolution have a direct effect on the hydrodynamics and the rate of sediment transport. The focus of this study is long wavelength ripples (LWRs) observed at two locations in the nearshore at roughly 3m water depth under combined current and wave conditions in Duck, North Carolina. The observed LWRs are straight-crested bedforms with wavelengths in the range of 30-75 cm. They occur during the build up of storms, when the incident wave direction is rapidly changing, possibly due to the migration of the center of a storm. A principal goal of the study is to test the maximum gross bedform-normal transport (mGBNT) hypothesis, which states that the orientation of ripples in directionally varying flows is such that the gross sediment transport normal to the ripple crest is maximized. Ripple wavelengths and orientation are measured from rotary fanbeam images and current and wave conditions are obtained from electromagnetic (EM) flowmeters and an offshore pressure gauge array. Tests of the mGBNT hypothesis in which the transport was calculated using a sediment transport model indicate that it is not a good predictor of LWR orientation. The observed LWR orientation seems to be tied to the incident wave direction, with an additional offset the sign of which depends on the sign of the longshore current.
\end{minipage}\end{minipage}

%-----------------------------------------------------------------
\vspace{5 mm} \begin{center}\rule{0.9\linewidth}{1pt}\end{center}
%-----------------------------------------------------------------
\begin{minipage}{\linewidth}\begin{center}\begin{minipage}{\linewidth}
  \abTitle{Does turbulence sound the same from over here, under there, everywhere?} \vspace{2 mm} \begin{center}
  \abSpeaker{Doris~Leong}{1}\abCoauthorO{Tetjana Ross}{1}  \vspace{2 mm}\begin{center}
  
  $\abAffilO{Department of Oceanography, Dalhousie University, Halifax, NS, B3H 4J1, Canada}{1}$

  \end{center}
  \vspace{2 mm}\abEmail{doris.leong@dal.ca}
  \end{center}\end{minipage}\end{center}
  \begin{center}\rule{0.70\linewidth}{0.5 pt}\end{center}
  \begin{minipage}{\linewidth}
\noindent Small-scale ocean turbulence can manifest as temperature and salinity microstructure that scatter sound. Acoustic measurements of turbulent fluctuations are translated into physical parameters using models that generally assume isotropic scattering, although there is no clear experimental evidence to support this. We investigate the existence of anisotropy in small-scale turbulence generated by internal waves. Strong anisotropy is predicted to noticeably shift the dissipative roll-off in the spectral frequency response of turbulence to lower frequencies. Observations of scatter from turbulence are made using a broadband acoustic system that is capable of sampling vertically or horizontally through the water column. The overall scatter shows statistical patterns in spectral shape that suggest the presence of anisotropy in either biological or physical scatter. Turbulence dissipation rates are estimated from acoustic inversions of spectra and yield no clear evidence of small-scale turbulence anisotropy.
\end{minipage}\end{minipage}

%-----------------------------------------------------------------
\vspace{5 mm} \begin{center}\rule{0.9\linewidth}{1pt}\end{center}
%-----------------------------------------------------------------
\begin{minipage}{\linewidth}\begin{center}\begin{minipage}{\linewidth}
  \abTitle{Sea level rise: A better understanding from new satellite measurements} \vspace{2 mm} \begin{center}
  \abSpeaker{Simon~Higginson}{1}  \vspace{2 mm}\begin{center}
  
  $\abAffilO{Department of Oceanography, Dalhousie University, Halifax, NS, B3H 4J1, Canada}{1}$

  \end{center}
  \vspace{2 mm}\abEmail{simon.higginson@dal.ca}
  \end{center}\end{minipage}\end{center}
  \begin{center}\rule{0.70\linewidth}{0.5 pt}\end{center}
  \begin{minipage}{\linewidth}
\noindent The geoid represents global mean sea level if the oceans were at rest. Mean dynamic topography (MDT) is the mean variation of the height of the ocean relative to the geoid, and can be related to the mean circulation by an assumption of geostrophy. Poor knowledge of the geoid has prevented direct measurement of MDT, and estimates have been based on indirect methods using hydrographic data. However the GRACE satellite gravity mission is providing detailed information on the geoid and its temporal change, leading to improved estimates of MDT. GRACE also provides the mass change of the ocean which, when combined with hydrographic data, provides valuable information on the relative contribution of ice melt and ocean warming to total sea level rise. Studies of the spatial pattern of sea level change have tended to simplify the ocean response. There is scope to introduce more realism, considering future changes to MDT and the circulation resulting from the predicted total sea level rise.
\end{minipage}\end{minipage}

%-----------------------------------------------------------------
\vspace{5 mm} \begin{center}\rule{0.9\linewidth}{1pt}\end{center}
%-----------------------------------------------------------------
\begin{minipage}{\linewidth}\begin{center}\begin{minipage}{\linewidth}
  \abTitle{Internal wave generation in the St. Lawrence Estuary} \vspace{2 mm} \begin{center}
  \abSpeaker{Clark~Richards}{1}\abCoauthorO{Dan Kelley}{1}  \vspace{2 mm}\begin{center}
  
  $\abAffilO{Department of Oceanography, Dalhousie University, Halifax, NS, B3H 4J1, Canada}{1}$

  \end{center}
  \vspace{2 mm}\abEmail{clark.richards@dal.ca}
  \end{center}\end{minipage}\end{center}
  \begin{center}\rule{0.70\linewidth}{0.5 pt}\end{center}
  \begin{minipage}{\linewidth}
\noindent Mixing in coastal environments is a process affecting many branches of oceanography; it contributes to fluxes of heat and salt, and redistributes chemical and biological tracers. Internal waves are a common feature in the stratified ocean, and are believed to be an important contributor to mixing. Recent and ongoing studies in the St. Lawrence Estuary have identified regions of internal wave propagation and dissipation, but to date little has been done to examine the generation phase. Fieldwork performed in the summer of 2008 identified a potential source region for internal waves, and data were collected to characterize the physical properties of the water column and tidal flow. This presentation will focus on shipboard and moored ADCP time series, echosounder transects, and CTD data as they relate to several different theories for wave generation.
\end{minipage}\end{minipage}

%-----------------------------------------------------------------
\vspace{5 mm} \begin{center}\rule{0.9\linewidth}{1pt}\end{center}
%-----------------------------------------------------------------
\begin{minipage}{\linewidth}\begin{center}\begin{minipage}{\linewidth}
  \abTitle{The Madden-Julian Oscillation and Local and Remote Forcing of the Ocean} \vspace{2 mm} \begin{center}
  \abSpeaker{Eric~Oliver}{1}\abCoauthorO{Keith Thompson}{1}  \vspace{2 mm}\begin{center}
  
  $\abAffilO{Department of Oceanography, Dalhousie University, Halifax, NS, B3H 4J1, Canada}{1}$

  \end{center}
  \vspace{2 mm}\abEmail{eric.oliver@phys.ocean.dal.ca}
  \end{center}\end{minipage}\end{center}
  \begin{center}\rule{0.70\linewidth}{0.5 pt}\end{center}
  \begin{minipage}{\linewidth}
\noindent The Madden-Julian Oscillation (MJO) is the dominant mode of atmospheric variability in the tropical atmosphere on intraseasonal timescales (i.e., weeks to seasons).  It is an eastward-propagating phenomenon with clear expressions in outgoing longwave radiation, precipitation and zonal wind stress over the tropical oceans.  The MJO has the potential to help bridge the gap between between extended-range weather forecasts and seasonal climate forecasts of both the atmosphere and ocean.  Observational and modeling studies have shown that the MJO can drive variability in the tropical ocean through local heat and momentum fluxes. In this study we examine the connection between sea level and the MJO on a global scale. We first identify regions exhibiting a significant (both statistical and practical) relationship between sea level and the MJO. The first region consists of the equatorial Pacific and western coast zones of North and South America. Consistent with previous studies, we identify wind-driven equatorially trapped Kelvin waves propagating eastward along the equatorial Pacific and then poleward along the coastal trapped waveguides of North and South America.  The second region includes the shallow waters of the Gulf of Carpentaria along the north coast of Australia and the adjacent Arafura and Timor Seas. Sea level set up by onshore winds is shown to be the dominant physical process. Finally, the northeastern Indian Ocean is shown to be a complex region involving a combination of equatorially trapped Kelvin waves, coastal trapped waves and westward propagating Rossby waves exhibiting characteristics of both local and remote forcing. The implications for deep and coastal ocean forecasting are discussed.
\end{minipage}\end{minipage}

%-----------------------------------------------------------------
\vspace{5 mm} \begin{center}\rule{0.9\linewidth}{1pt}\end{center}
%-----------------------------------------------------------------
\begin{minipage}{\linewidth}\begin{center}\begin{minipage}{\linewidth}
  \abTitle{Temperature, timing and growth: implications for outbreaks of an introduced species} \vspace{2 mm} \begin{center}
  \abSpeaker{Megan~Saunders}{1}\abCoauthorO{Anna Metaxas}{1}\abCoauthorO{Ram�Á�n Filgueira}{1}  \vspace{2 mm}\begin{center}
  
  $\abAffilO{Department of Oceanography, Dalhousie University, Halifax, NS, B3H 4J1, Canada}{1}$

  \end{center}
  \vspace{2 mm}\abEmail{msaunders@dal.ca}
  \end{center}\end{minipage}\end{center}
  \begin{center}\rule{0.70\linewidth}{0.5 pt}\end{center}
  \begin{minipage}{\linewidth}
\noindent Outbreaks of the invasive bryozoan \sl Membranipora membranacea \rm in the western Atlantic facilitate the invasion of other algae by defoliating kelp beds. To examine the effect of temperature on the \sl M. membranacea \rm population, we constructed an individual-based population model, which successfully simulated the timing of onset of settlement, number of adult colonies, maximum colony diameter, and relative interannual patterns in abundance. We used the model to examine the relative effect on the population of varying temperature by -2 to +2$^{\circ}$C day$^{-1}$. Increasing daily temperature by 2$^{\circ}$C caused the population to occur 1 month earlier in the season, and resulted in a 100 fold increase in abundance. Changes in winter and summer temperature had the most pronounced effects on the timing and abundance of the population, respectively. Our results suggest that outbreaks of this species will be more pronounced if temperature increases as a result of climate change. 
\end{minipage}\end{minipage}

%-----------------------------------------------------------------
\vspace{5 mm} \begin{center}\rule{0.9\linewidth}{1pt}\end{center}
%-----------------------------------------------------------------
\begin{minipage}{\linewidth}\begin{center}\begin{minipage}{\linewidth}
  \abTitle{Influence of density-dependent food consumption, foraging and stacking behaviour on the growth rate of the Northern abalone, Haliotis kamtschatkana} \vspace{2 mm} \begin{center}
  \abSpeaker{Michelle~Lloyd}{1}\abCoauthorO{Amanda Bates}{1}  \vspace{2 mm}\begin{center}
  
  $\abAffilO{Bamfield Marine Sciences Centre, Bamfield, BC, V0R 1B0, Canada}{1}$

  \end{center}
  \vspace{2 mm}\abEmail{michelle.lloyd@dal.ca}
  \end{center}\end{minipage}\end{center}
  \begin{center}\rule{0.70\linewidth}{0.5 pt}\end{center}
  \begin{minipage}{\linewidth}
\noindent Growth of abalone in the wild and hatchery is density-dependent in response to intraspecific competition for food and/or space. To determine if a candidate aquaculture species, Haliotis kamtschatkana, exhibits density-dependent growth we raised animals at three density levels and two food treatments: unlimited (ad libitum) and rationed (individual portions were the same among density treatments). We also tested for differences in food consumption, foraging patterns and stacking behaviour among the density levels. We observed density-dependent growth in the rationed treatments, indicating that relatively high growth rates at lower densities are driven, in part, by factors other than differences in food consumption.However, overall the quantity of food consumed related directly to growth; treatments fed ad libitum had higher growth rates. Furthermore, even when food was provided in excess, foraging was restricted to $\sim$2 h after sunset in all treatments and the amount consumed per abalone was significantly lower at high densities. This is probably because high density animals could not access the food provided: fewer were observed foraging and they had to move from prominent stacks. Our results indicate that both temporal and spatial access to food are critical and that managers can observe foraging and stacking by abalone in tanks to determine if a specific design will limit food consumption, and ultimately growth.
\end{minipage}\end{minipage}

%-----------------------------------------------------------------
\vspace{5 mm} \begin{center}\rule{0.9\linewidth}{1pt}\end{center}
%-----------------------------------------------------------------
\begin{minipage}{\linewidth}\begin{center}\begin{minipage}{\linewidth}
  \abTitle{Ships voluntarily alter course to protect endangered whales} \vspace{2 mm} \begin{center}
  \abSpeaker{Angelia~S.M.~Vanderlaan}{1}\abCoauthorO{Christopher T. Taggart}{1}  \vspace{2 mm}\begin{center}
  
  $\abAffilO{Department of Oceanography, Dalhousie University, Halifax, NS, B3H 4J1, Canada}{1}$

  \end{center}
  \vspace{2 mm}\abEmail{avanderl@phys.ocean.dal.ca}
  \end{center}\end{minipage}\end{center}
  \begin{center}\rule{0.70\linewidth}{0.5 pt}\end{center}
  \begin{minipage}{\linewidth}
\noindent Ocean-going vessels pose a threat to large whales worldwide and are responsible for the majority of deaths diagnosed among endangered North Atlantic right whales (\sl Eubalaena glacialis\rm). Various conservation measures, including vessel re-routing and vessel-speed restrictions, have been implemented to reduce vessel-strike mortality in this species. We initiated the Vessel Avoidance & Conservation Area Transit Experiment (VACATE) to evaluate the efficacy of a voluntary and seasonal Area to be Avoided (ATBA) in reducing the risk of lethal vessel-strikes. The ATBA was adopted by the International Maritime Organization (IMO) for the Roseway Basin region of the Scotian Shelf in 2008. The effectiveness of this vessel-avoidance scheme in reducing risk without the imposition of vessel-speed restrictions is entirely dependent on vessel-operator compliance.  Using a network of Automatic Identification System receivers we collected static, dynamic, and voyage-related vessel data in the Roseway Basin region, both pre- and post-implementation of the ATBA.  Our analyses show that semimonthly estimates of vessel-operator voluntary compliance range from 57\% to 87\%, and stabilised at 71\% within the first 5 months of implementation.  Using pre- and post-implementation vessel-navigation and speed data, along with right whale sightings per unit effort data, we estimate an 82\% reduction in the risk of lethal vessel-strikes to right whales that is due to vessel-operator compliance. The high level of compliance achieved with this voluntary conservation initiative is likely due to the ATBA being adopted by the IMO. Through VACATE we demonstrate that the international shipping industry is able and willing to voluntarily alter course to protect endangered whales.  
\end{minipage}\end{minipage}

%-----------------------------------------------------------------
\vspace{5 mm} \begin{center}\rule{0.9\linewidth}{1pt}\end{center}
%-----------------------------------------------------------------
\begin{minipage}{\linewidth}\begin{center}\begin{minipage}{\linewidth}
  \abTitle{Development of a nested-grid shelf circulation model using OPA for the eastern Canadian shelf} \vspace{2 mm} \begin{center}
  \abSpeaker{Jorge~R.~Urrego-Blanco}{1}\abCoauthorO{Jinyu Sheng}{1}  \vspace{2 mm}\begin{center}
  
  $\abAffilO{Department of Oceanography, Dalhousie University, Halifax, NS, B3H 4J1, Canada}{1}$

  \end{center}
  \vspace{2 mm}\abEmail{jorge.urrego.blanco@dal.ca}
  \end{center}\end{minipage}\end{center}
  \begin{center}\rule{0.70\linewidth}{0.5 pt}\end{center}
  \begin{minipage}{\linewidth}
\noindent As a first step of developing a nested-grid circulation model for the eastern Canadian shelf, we constructed a coarse-grid (1/4$^\circ$) northwest Atlantic circulation model using the ocean general circulation model known as OPA (Oc\'{e}an PArall\'{e}lis\'{e}). The model domain covers the area between 32$^\circ$W and 81$^\circ$W and between 33$^\circ$N and 57$^\circ$N. This model was used to simulate the 3-D circulation from 1990 to 1999 in this study. The model was forced by atmospheric reanalysis fields produced by Large and Yeager (2004) and monthly mean climatologies of temperature and salinity produced by Geshelin et al. (1999). Three different numerical experiments were conducted to examine the model performance in simulating large-scale circulation over the study region. These three experiments are: a) a fully prognostic run without data assimilation; b) a run using the spectral nudging method (Thompson et al. 2006); and c) a run using the semi-prognostic method (Sheng et al. 2001). In the first experiment no hydrographical assimilation is made and model results in this experiment demonstrate significant model drift and unrealistic circulation features for a multi-year model integration. For the spectral nudging experiment the model drift in TS fields is significantly reduced and the hydrographical seasonal cycle is well reproduced by the model as expected. However, tracer variability in this run is strongly damped and the eddy field is less free to evolve in the model. In the semi-prognostic run a correction (or assimilation) term is introduced in the model through the hydrostatic equation while leaving the model tracer equations to be fully prognostic. Model results demonstrate that the semi-prognostic method not only reduces model drift but also improves the flow field simulation. This method however still damps the mesoscale eddy field. Future work will include the use of a smoothed semi-prognostic method which allows the mesoscale eddy field to be more realistically reproduced.
\end{minipage}\end{minipage}

%-----------------------------------------------------------------
\vspace{5 mm} \begin{center}\rule{0.9\linewidth}{1pt}\end{center}
%-----------------------------------------------------------------
\begin{minipage}{\linewidth}\begin{center}\begin{minipage}{\linewidth}
  \abTitle{Numerical and observational study of circulation in the Intra-Americas Sea: connection between Gulf of Mexico Loop Current intrusion and throughflow transport variability} \vspace{2 mm} \begin{center}
  \abSpeaker{Yuehua~Lin}{1}\abCoauthorO{Richard Greatbatch}{2}\abCoauthorO{Jinyu Sheng}{1}  \vspace{2 mm}\begin{center}
  
  $\abAffilO{Department of Oceanography, Dalhousie University, Halifax, NS, B3H 4J1, Canada}{1}$

  
  $\abAffilO{Leibniz Institute of Marine Sciences at Kiel University (IFM-GEOMAR), 24105 Kiel, Germany}{2}$

  \end{center}
  \vspace{2 mm}\abEmail{Yuehua.Lin@phys.ocean.dal.ca}
  \end{center}\end{minipage}\end{center}
  \begin{center}\rule{0.70\linewidth}{0.5 pt}\end{center}
  \begin{minipage}{\linewidth}
\noindent Significant correlation between temporal variations of sea surface height anomalies in the Loop Current region and transport variations through the Yucatan Channel in the Intra-Americas Sea is found based on the analysis of numerical model results and satellite-altimeter data. Transport in the model is found to be a minimum when the Loop Current intrudes strongly into the Gulf of Mexico, typically just before a ring is shed, and to be a maximum during the next growth phase in association with the build up of warm water off the northwest coast of Cuba. Numerical experiments show that the transport variations result from the interaction between the density anomalies associated with Loop Current intrusion and the variable bottom topography. A proxy for low-frequency transport variations through the Yucatan Channel is then proposed, which compares well with the 2-year transport estimates for the Yucatan Channel during the CANEK program (10 September 1999 to 31 May 2001). A 10-year comparison between the transport proxy and the cable data sheds light on the influence of Loop Current intrusion on the Florida Current between Florida and the Bahamas.
\end{minipage}\end{minipage}

%-----------------------------------------------------------------
\vspace{5 mm} \begin{center}\rule{0.9\linewidth}{1pt}\end{center}
%-----------------------------------------------------------------
\begin{minipage}{\linewidth}\begin{center}\begin{minipage}{\linewidth}
  \abTitle{Numerical Study of Tidal Circulation in Jiaozhou Bay and Adjacent Coastal Waters Using a High-resolution, Three-dimensional Circulation Model} \vspace{2 mm} \begin{center}
  \abSpeaker{Shiliang~Shan}{1,2}\abCoauthorO{Huaming Yu}{2}\abCoauthorO{Xueen Chen}{2}\abCoauthorO{Jinrui Chen}{2}  \vspace{2 mm}\begin{center}
  
  $\abAffilO{Department of Oceanography, Dalhousie University, Halifax, NS, B3H 4J1, Canada}{1}$

  
  $\abAffilO{College of Physical and Environmental Oceanography, Ocean University of China, Qingdao 266100, P. R. China}{2}$

  \end{center}
  \vspace{2 mm}\abEmail{sshan@phys.ocean.dal.ca}
  \end{center}\end{minipage}\end{center}
  \begin{center}\rule{0.70\linewidth}{0.5 pt}\end{center}
  \begin{minipage}{\linewidth}
\noindent A high-resolution, three-dimensional coastal circulation model was constructed for Jiaozhou Bay and adjacent coastal waters using the finite-volume method (FVM). The main advantage of the FVM is that complex coastline and irregular topography can easily be represented in the model by unstructured triangular grid. The coastal circulation model has a fine-resolution of $\sim$30 m for harbours and Qingdao Olympic Sailing Center and uses the wet/dry lateral boundaries for the inter-tidal zone. The model results demonstrate that tides in Jiaozhou Bay are mainly semi-diurnal. Outside Jiaozhou Bay, tidal waves propagate from the northeast to southwest with a cyclonic rotation. As approaching the mouth of Jiaozhou Bay, tidal currents bifurcate: with one continually propagates to southwest along the coastline, the other one propagates into inner Jiaozhou Bay with increasing amplitude. Near the mouth of Jiaozhou Bay, Eulerian residual currents have a multi-eddy structure, with surface residual tidal currents greater than the bottom currents. Tidal energy propagates from the northeast to southwest outside Jiaozhou Bay. Near the mouth of Jiaozhou Bay part of tidal energy transmits to southwest along the coastline, the other part of energy converges at mouth of Jiaozhou Bay, and then diverges to the inward Bay. A numerical dye release experiment demonstrates that the mouth of Jiaozhou Bay is an active zone of water exchange.
\end{minipage}\end{minipage}

%-----------------------------------------------------------------
\vspace{5 mm} \begin{center}\rule{0.9\linewidth}{1pt}\end{center}
%-----------------------------------------------------------------
\begin{minipage}{\linewidth}\begin{center}\begin{minipage}{\linewidth}
  \abTitle{Groundline profiles on the Bay of Fundy lobster gear as a threat to North Atlantic right whales} \vspace{2 mm} \begin{center}
  \abSpeaker{Sean~Brillant}{1,2}\abCoauthorO{Ed Trippel}{3}  \vspace{2 mm}\begin{center}
  
  $\abAffilO{Department of Oceanography, Dalhousie University, Halifax, NS, B3H 4J1, Canada}{1}$

  
  $\abAffilO{WWF-Canada, Halifax NS}{2}$

  
  $\abAffilO{Fisheries and Oceans Canada, St. Andrews Biological Station, St. Andrews NB}{3}$

  \end{center}
  \vspace{2 mm}\abEmail{sbrillant@wwfcanada.org}
  \end{center}\end{minipage}\end{center}
  \begin{center}\rule{0.70\linewidth}{0.5 pt}\end{center}
  \begin{minipage}{\linewidth}
\noindent Conservation of North Atlantic right whales (Eubalaena glacialis) requires mortalities caused by human activities to be significantly reduced. Entanglement in fishing gear is considered an important cause of mortality and one that is underestimated. In order to reduce this risk, we must know where the whales are, where the gear is and the probability of lethal outcome if an encounter between the two occurs. An important component of this concerns the groundline, a rope used to attach traps (e.g. crab, lobster) in a series (trawl). A common assumption is that groundlines form arches in the water column and are a threat to whales. Many fishermen have challenged this interpretation. This research measured the elevations of groundlines in the Bay of Fundy and evaluated several factors that could influence them. Sensors were attached to nineteen groundlines on seven different lobster trawls that were being actively fished in the Bay of Fundy and elevations were recorded for at least two days. Results suggest that groundlines are within 3 m of the bottom most of the time, but groundlines of poorly set trawls may reach as high as 7 m. Although many factors did not influence elevations (e.g. water depth), some did (e.g. current velocity) and it was concluded that as a result of these influencing factors, fishermen in the Bay of Fundy can help ensure that their groundlines remain low, reducing the risk to right whales.
\end{minipage}\end{minipage}

%-----------------------------------------------------------------
\vspace{5 mm} \begin{center}\rule{0.9\linewidth}{1pt}\end{center}
%-----------------------------------------------------------------
\begin{minipage}{\linewidth}\begin{center}\begin{minipage}{\linewidth}
  \abTitle{Methods: describing phytoplankton physiological variability} \vspace{2 mm} \begin{center}
  \abSpeaker{Adam~J.~Comeau}{1}\abCoauthorO{John Cullen}{1}  \vspace{2 mm}\begin{center}
  
  $\abAffilO{Department of Oceanography, Dalhousie University, Halifax, NS, B3H 4J1, Canada}{1}$

  \end{center}
  \vspace{2 mm}\abEmail{adam.comeau@dal.ca}
  \end{center}\end{minipage}\end{center}
  \begin{center}\rule{0.70\linewidth}{0.5 pt}\end{center}
  \begin{minipage}{\linewidth}
\noindent By understanding factors that influence parameters related to photosynthesis, better estimates of primary productivity and particle dynamics can be obtained. We describe a new method to estimate photoacclimation, a physiological process that influences both photosynthesis vs. irradiance (\sl P \rm vs. \sl E\rm) parameters and chemical composition of phytoplankton, based on profiles of \sl in situ \rm fluorescence and irradiance and apply it to examine the variability of phytoplankton photoacclimation in relation to environmental variables.
Profiles of \sl in situ \rm chlorophyll fluorescence have been routinely measured during oceanographic surveys for several decades. Near surface decreases of fluorescence yield, chlorophyll fluorescence normalized to some measure of phytoplankton biomass, are commonly observed during daytime profiles. This decrease in fluorescence is due to physiological processes, activated in high irradiance, which act to dissipate light energy absorbed by phytoplankton. Lab studies show that the irradiance at which this quenching of fluorescence yield begins is related to a parameter used to estimate primary productivity. With the simple requirements of irradiance and fluorescence yield profiles, this method can be applied to many existing datasets. Examining variations of the light level where fluorescence quenching begins in response to environmental variables such as average light in the mixed layer, will provide new information on how phytoplankton acclimate to their environment.
\end{minipage}\end{minipage}

%-----------------------------------------------------------------
\vspace{5 mm} \begin{center}\rule{0.9\linewidth}{1pt}\end{center}
%-----------------------------------------------------------------
\begin{minipage}{\linewidth}\begin{center}\begin{minipage}{\linewidth}
  \abTitle{A Phytoplankton Pigment Extraction Protocol for Marine Sediments} \vspace{2 mm} \begin{center}
  \abSpeaker{Lisa~Delaney}{1}\abCoauthorO{Marlon Lewis}{1}\abCoauthorO{Markus Kienast}{1}  \vspace{2 mm}\begin{center}
  
  $\abAffilO{Department of Oceanography, Dalhousie University, Halifax, NS, B3H 4J1, Canada}{1}$

  \end{center}
  \vspace{2 mm}\abEmail{lisa.delaney@dal.ca}
  \end{center}\end{minipage}\end{center}
  \begin{center}\rule{0.70\linewidth}{0.5 pt}\end{center}
  \begin{minipage}{\linewidth}
\noindent Sediments were collected from Lunenburg Bay, NS and Saanich Inlet, BC, Canada. Pigments were extracted from these sediments using five different solvents. In addition, the effects of freeze-drying and different extraction times were tested. It was found that treatment time did not lead to significant changes in pigment concentration. Freeze-drying samples typically resulted in a significant decrease in pigment concentration. The general trend for solvent type showed that the more abrasive solvents extracted a greater concentration of pigment from Saanich Inlet sediments. From the Lunenburg Bay sediments, acetone-based mixtures extracted higher concentrations of pigment.

The effect of storage temperature was also assessed. Sediment samples were collected from the North Water (NOW) polynya, NU, Canada. Replicates were stored at one of four different temperatures for 22 months. It was found that samples stored below freezing yielded a significantly greater pigment concentration in almost every case.
\end{minipage}\end{minipage}

%-----------------------------------------------------------------
\vspace{5 mm} \begin{center}\rule{0.9\linewidth}{1pt}\end{center}
%-----------------------------------------------------------------
\begin{minipage}{\linewidth}\begin{center}\begin{minipage}{\linewidth}
  \abTitle{Physical forcing of space-time variation in the copepod prey field of North Atlantic right whales } \vspace{2 mm} \begin{center}
  \abSpeaker{Kimberley~T.~A.~Davies}{1}\abCoauthorO{Christopher T. Taggart}{1}  \vspace{2 mm}\begin{center}
  
  $\abAffilO{Department of Oceanography, Dalhousie University, Halifax, NS, B3H 4J1, Canada}{1}$

  \end{center}
  \vspace{2 mm}\abEmail{kim.davies@dal.ca}
  \end{center}\end{minipage}\end{center}
  \begin{center}\rule{0.70\linewidth}{0.5 pt}\end{center}
  \begin{minipage}{\linewidth}
\noindent Defining habitat critical for the survival of endangered species is the goal of many marine science initiatives. In the pelagic zone, feeding habitat boundaries are difficult to define statically in space because the prey-field is subject to advection and mixing by regional flow fields. Here we address this issue in relation to defining the critical feeding habitat of the endangered North Atlantic right whale. These whales feed on diapausing, lipid rich copepods that are aggregated in high concentrations, near bottom, in the deep (>100m) basins of the Scotia-Fundy region. We ask can variation in the spatial distribution of the right whale prey-field at short (tidal, diel) time scales be explained by variation in the current regime in Roseway Basin. Current speed and acoustic backscatter (zooplankton abundance) data were simultaneously collected using moorings at three locations in the Basin. Two moorings, fitted with upward looking Acoustic Doppler Current Profilers (ADCP; one 300 kHz and one 600 kHz), downward looking Aquadopp profilers, and CTDs (SBE-37), straddled a sloping region on the edge of the Basin where whales feed and were located within one tidal excursion from one another. The third mooring (upward looking 300 kHz ADCP) was located at the deepest portion of the Basin. The area was concurrently surveyed using a ship-mounted echo-sounder (Simrad, 120 kHz). We illustrate how zooplankton abundance varies in response to tidal advection of water masses and the residual circulation. 
\end{minipage}\end{minipage}


